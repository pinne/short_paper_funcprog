% The default of sigplan-proc-varsize is 9pt, indented paragraphs (acm style)
% For Sensys or other 10pt conference, use the 10pt option
%\documentclass{sigplan-proc-varsize}
% options:
%\documentclass[9pt]{sigplan-proc-varsize}
%\documentclass[10pt]{sigplan-proc-varsize}
\documentclass[10pt]{sensys-abstract}
%\documentclass[noindentedparagraphs]{sigplan-proc-varsize}

% % hack to avoid the ugly ACM paragraph definition
% % => can't leave blank line after this
% (remove comment for this hack)
%\renewcommand{\paragraph}[1]{\vskip 6pt\noindent\textbf{#1 }}

\usepackage{graphicx}
\usepackage{url}
\usepackage[T1]{fontenc}
\usepackage[utf8]{inputenc}
\usepackage[english]{babel}
\usepackage{blindtext}

\usepackage{etoolbox}
\makeatletter
\patchcmd{\maketitle}{\@copyrightspace}{}{}{}
\makeatother

\usepackage{times}
\usepackage{fancyhdr}
\usepackage{booktabs}
\DeclareGraphicsExtensions{.png}

\numberofauthors{2}

\author{
%
% The command \alignauthor (no curly braces needed) should
% precede each author name, affiliation/snail-mail address and
% e-mail address. Additionally, tag each line of
% affiliation/address with \affaddr, and tag the
%% e-mail address with \email.
\alignauthor Simon Kers\\
   \email{\url{skers@kth.se}}\\
   \affaddr{School of Technology and Health}\\
   \affaddr{KTH Royal Institute of Technology}\\
   \affaddr{February 2013}\\
\alignauthor David Wikmans\\
   \email{\url{wikmans@kth.se}}\\
   \affaddr{School of Technology and Health}\\
   \affaddr{KTH Royal Institute of Technology}\\
   \affaddr{February 2013}\\
}

\title{Centralized and distributed testing}

%Which methodologies and technologies are there and how are they used to test centralized and distributed systems behaviour and operation respectively"

%\conferenceinfo{SenSys'07,} {November 6--9, 2007, Sydney, Australia.}
\CopyrightYear{2012}
%\crdata{} % not submitted
%\crdata{1-59593-763-6/07/0011}

\begin{document}

\maketitle

\begin{abstract}
\noindent
%three sentences, not paragraphs.
%state the problem
This paper reviews some select methodologies and techniques used for testing the behaviour and reliability of software applications, and how testing differs between distributed and centralized systems.
%describe why this is a problem
Application operability, uptime and availability are concerns for large systems and services but testing these areas can be tricky.
%state the solution
In the software industry testing is not only necessary but common practice and with the emergance of distributed systems and the internet, testing faces new challenges.
%state results
By reducing dependencies\footnote{Low coupling} of the distributed parts of the system and purposely subjecting the network to high load, the distributed system can be hardened and better protected against bugs.

\end{abstract}

% A category with the (minimum) three required fields
% http://www.acm.org/about/class/ccs98-html
%\category{X.0}{}{}
%A category including the fourth, optional field follows...
%\category{}{}{}[complexity measures, performance measures]

%\terms{} % http://www.sheridanprinting.com/typedept/generalterms.htm

%\keywords{}

\section{Introduction}
  \label{sec:intro}
\noindent
Software testing has been a part of software-development as long as there have been computers. To assess the quality of computer software testing is very important. Testing typically takes upwards 40\% -- 50\% of the development time, and even more for systems that require higher levels of reliability. \cite{testing_techniques}

Software testing is divided into two schools of thought. Static testing which is to analyze the source code by inspections, peer reviews and walkthroughs. Dynamic testing where the application is executed and the behaviour of the program, or rather, the systems response is tested.

Testing can be done after the development process by an independent group of testers, which is the traditional or \emph{Waterfall} approach. Writing tests before the development process begins is known as \emph{Test Driven Development}.

\section{Evaluation}

\subsection{Centralized systems} \noindent

\emph{Unit tests} are tests for select code segments. Normally these tests are on a functional level. In object-oriented languages unit tests means tests for classes. 

The \emph{box approach} is where you put something in the box and then return a result. A black-box is where you do not care about the inner workings of the box (implementation) only the return value. A white or clear-box is when you test the interal structure of the object. A gray-box test is a combination of both.

Another is the \emph{bottom-up top-down approach}, where bottom-up means writing tests for each component and top-down only the functions of the high level objects.

Mentioned techniques are not exclusive to centralized systems.
\subsection{Distributed systems}

\subsubsection{States}\noindent
 %kanske flytta till introduction, state space innebär att det uppstår fler möjliga tillstånd (som kan vara buggiga) i distr. applikationer.
Distributed applications, effectively increases the number of possible states the system can have. One meassure against this is employing designs which reduce dependencies between nodes, that the state of one node affects the state of another as little as possible.

Distributed systems rely on communication between nodes, this could lead to an increased number of states the system as a whole can take, increasing the possibility of bugs.\cite{Rikitake:2011:2034654} %kolla denna ref!:(

%\subsubsection{Network}\noindent
%Simulate network degradation like latency, packet loss and bandwidth restriction. kanske kan vara ett ämne

\subsubsection{Chaos monkey}\noindent
When testing distributed systems a new class of software bugs emerge which conventional testing procedures may not catch. One inconvenient property of such bugs are infrequent occurrences and are difficult to reproduce, especially with code not in production\footnote{Code that is currently in development}.

One counterintuitive approach would be Netflix \emph{Chaos Monkey} implementation, a disaster testing system, which operates on Amazons cloud computing platform. It randomly causes failures of the virtual machine instances, know as \emph{Auto Scaling Groups}.\cite{netflix_chaosmonkey}

This puts the distributed systems resiliency at test and enforces security measure against netsplits, erratic behavior and failure caused by high latency and bugs of such distributed nature. It enables software developers to discover rarely occuring bugs and resolve them before the code goes live.

\section{Conclusions}
\noindent
A whole set of new problems arises when developing distributed systems, the proneness to error that occurs in centralized software are still there in distributed systems. One way of decreasing the need for distributed software testing is to reduce coupling between the different nodes. This isolates the test cases to a local level and avoids some aspects of distributed testing procedures and simplifies code by treating the nodes as resources.

External occurences such as power outages and network failure are hard to predict, therefore new ways of simulating failure and load has gained ground. Distributed system often grow large and produce detailed logs of programmatic operation and network status, analyzing large quantities of log data becomes more important as these systems grow and becomes more prevalent.

Though tests serve an important purpose, they are only as good as their implementation and can not guarantee a fault-less system. Testing distributed aspects will become an important part of developing large scale systems.

\newpage
%\bibliographystyle{acmtrans}
\bibliographystyle{plain}
%\bibliographystyle{abbrv}
\bibliography{func_oop_testing} % from Github pinne funcoop
% You must have a proper ".bib" file
%  and remember to run:
% latex bibtex latex latex
% to resolve all references
%
% ACM needs 'a single self-contained file'!

%APPENDICES are optional
\balancecolumns
\end{document}
\appendix
%Appendix A
\subsection{Additional Authors}
This section is inserted by \LaTeX; you do not insert it.
You just add the names and information in the
\texttt{{\char'134}additionalauthors} command at the start
of the document.
\subsection{References}
Generated by bibtex from your ~.bib file.  Run latex,
then bibtex, then latex twice (to resolve references)
to create the ~.bbl file.  Insert that ~.bbl file into
the .tex source file and comment out
the command \texttt{{\char'134}thebibliography}.
% This next section command marks the start of
% Appendix B, and does not continue the present hierarchy
%\balancecolumns % GM MARCH 2007
% That's all folks!
\end{document}
